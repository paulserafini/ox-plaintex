\input eplain.tex

\def\begincenter{%
\par
\begingroup
\leftskip=0pt plus 1fil
\rightskip=\leftskip
\parindent=0pt
\parfillskip=0pt
}
\def\endcenter{%
\par
\endgroup
}
\long\def\centerpar#1{\begincenter#1\endcenter}

\font\eighteenrm= cmr10 at18pt
\font\fourteenrm= cmr10 at 14pt%
\font\twelverm= cmr10 at12pt


\def\beginquote{\begingroup\par\narrower\smallskip\noindent}
\def\endquote{\smallskip\endgroup\noindent}


\def\toprule{\noalign{\smallskip\hrule height 1pt}}
\def\midrule{\noalign{\vskip 0.25em \hrule height 0.5pt}}
\def\bottomrule{\noalign{\vskip 0.25em \hrule height 1pt}}
\def\tstrut{\vrule height 12pt depth3pt width0pt}

\newbox\TestBox
\def\sout#1{\setbox\TestBox=\hbox{#1}%
\leavevmode\rlap{\vrule height 2.5pt depth-1.75pt width\wd\TestBox}%
\box\TestBox\ }

\def\newpage{\vfill\break}

\def\title#1{\def\title{#1}}
\def\author#1{\def\author{#1}}

\def\maketitle{%
{%
  \nopagenumbers
  \baselineskip = 25pt
  \topglue 0pt plus 1fill
  \centerpar{\eighteenrm\title}
  \centerline{\twelverm\author}
  \centerline{\twelverm\today}
  \tenrm\newpage
}
\pageno -2
}

\def\keywords#1{\def\keywords{#1}}
\def\abstract#1{\def\abstract{#1}}
\def\makeabstract{%
  \centerline {\bf Abstract}
  {\narrower\smallskip\noindent \abstract{}}
  {\narrower\smallskip\noindent \keywords{}\par}
  \newpage
}

\newif\ifrunfoot
\newif\ifrunhead
\newcount\chapterNumber
\def\chapter #1
\par{%
  \newpage
  \eject\topglue 1in

  % TeXbook, p. 253
  \def\rightheadline{\tenrm\hfil #1\enskip\folio}
  \def\leftheadline{\tenrm\folio\enskip #1\hfil}

  % A TEX Primer for Scientists, p. 118
  \runheadfalse
  \headline={\ifrunhead
    \ifodd\pageno\rightheadline \else\leftheadline\fi
    \else \hfil \global\runheadtrue \fi}

  % A TEX Primer for Scientists, p. 118
  \runfootfalse
  \footline={\ifrunfoot
    \hfil
    \else \hfil\rm\folio\hfil \global\runfoottrue \fi} % turned bold at on point, hence \rm
  
  \advance \chapterNumber by 1
  \ifnum\chapterNumber=1 \pageno 1 \fi
  \sectionNumber=0
  % \tableNumber=0
  % \figureNumber=0
  % \scriptNumber=0

  {\eighteenrm\noindent\the\chapterNumber\ #1}
  \writenumberedtocentry{chapter}{#1}{\the\chapterNumber}

  \bigskip\noindent
}

\newcount\sectionNumber
\def\section #1
\par{%
  \bigskip
  \advance \sectionNumber by 1
  \subsectioncount=0
  {\twelverm\noindent\the\chapterNumber.\the\sectionNumber\ #1}
  \xdef\ID{\the\chapterNumber.\the\sectionNumber}
  \writenumberedtocentry{section}{#1}{\ID}

  \smallskip\noindent
}

\newcount\subsectioncount
\def\subsection #1
\par{%
  \medskip
  \advance \subsectioncount by 1
  {\noindent\bf\the\chapterNumber.\the\sectionNumber.\the\subsectioncount\ #1}
  \xdef\ID{\the\chapterNumber.\the\sectionNumber.\the\subsectioncount}
  \writenumberedtocentry{subsection}{#1}{\ID}

  \noindent
}

\def\tocchapterentry#1#2#3{\line{\bf #2 \enskip #1 \dotfill\ #3}}%
\def\tocsectionentry#1#2#3{\line{\quad #2 \enskip #1 \dotfill\ #3}}%
\def\tocsubsectionentry#1#2#3{\line{\qquad #2 \enskip #1 \dotfill\ #3}}%

\newcount\tablecount
\def\label#1{\advance \tableNumber by 1 \definexref{#1}{\the\tableNumber}{table}}